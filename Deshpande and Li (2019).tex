\documentclass{article}
\usepackage[utf8]{inputenc}

%\usepackage[left=3cm, right=2.5cm, top=2.5cm, bottom=2.5cm]{geometry}e}
\usepackage[spanish,english]{babel}
\usepackage{apacite}
\usepackage[round]{natbib}
\usepackage{hyperref}
\usepackage[margin = 1in, top=2cm]{geometry}% Margins
\setlength{\parindent}{2em}
\setlength{\parskip}{0.2em}
\usepackage{setspace} % Setting the spacing between lines
\usepackage{hyperref} % To create hyperlinks within the document
\spacing{1.15}

\usepackage[round]{natbib}
%\bibliographystyle{plainnat}
\bibliographystyle{apacite}

\title{Outline of Deshpande and Li (2019)}
\author{Giselle Labrador Badia}
\date{October 2021}

\begin{document}

\maketitle

The following is an outline of the introduction of \cite{deshpande2019screened}

\begin{enumerate}

\item Introduction:  Large and rapidly expanding disability programs aim to provide benefits to individuals with severe disabilities and in need of assistance.

\item Motivation: The effect of application costs of  Disability insurance (DI) has an ambiguous effect on the targeting of the programs.

    \item Research question and debate: The authors address how application cost affect targeting of disability programs, which the theoretical and behavioral economics literature argues yields improvement and loss of efficiency respectively (see \cite{nichols1982targeting} and \cite{bertrand2004behavioral})
    
    \item State Goal and Contribution to the literature 1: Provide the first empirical analysis of the effect of screening cost on targeting of disability programs using variation in the timing of closings of Social Security Administration (SSA).
    
    \item Findings 1: The authors find that the field office closings reduce the number of disability applications and recipients and disproportionately discourage applications of individuals who would have been admitted otherwise.
    
    \item Findings 2: Using instrumental variables, they estimate that the channels through which closing reduces applications are mainly increased congestion at offices (54 percent) and cost of switching offices (42 percent), while increased driving costs have a smaller effect (4 percent).
    
    \item Contributions to the literature 2: Advance the literature of screening and targeting efficiency in the context of SSA office closing, and contributes to the debate by siding with the literature that posed a loss of efficiency in contrast to \cite{nichols1982targeting} hypothesis. 
    
    \item Normative Findings: Through a cost-benefit analysis the paper finds that closings have an adverse effect on social welfare and that this effect is milder the stricter the severity standards are and flips when only the most severe are considered deserving.
    
    \item Roadmap
    
\end{enumerate}





\bibliography{references}

\end{document}

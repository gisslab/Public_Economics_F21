\documentclass{article}
\usepackage[utf8]{inputenc}

%\usepackage[left=3cm, right=2.5cm, top=2.5cm, bottom=2.5cm]{geometry}e}
\usepackage[spanish,english]{babel}
\usepackage{apacite}
\usepackage[round]{natbib}
\usepackage{hyperref}
\usepackage[margin = 1.3in, top=2cm]{geometry}% Margins
\setlength{\parindent}{2.0em}
\setlength{\parskip}{0.3em}
\usepackage{setspace} % Setting the spacing between lines
\usepackage{hyperref} % To create hyperlinks within the document
\spacing{1.25}

\usepackage[round]{natbib}
%\bibliographystyle{plainnat}
\bibliographystyle{apacite}

\title{Letter to editor  about 'Beyond health: Non-health risk and the value of disability insurance'}
%\author{Giselle Labrador Badia}
\date{September 2021}

\begin{document}

\maketitle

Dear Corina, 

\hspace{2cm}

The paper 'Beyond health: Non-health risk and the value of disability insurance' is thoughtful and carefully written. It is clear what the authors are set to do and what their findings and contributions are.  The paper motivates the research question and the approach very wisely: health does not tell the whole story about the value of disability programs. 

I did not find that the keys observations about the recipients and non-recipients given their health were published before. I consider that although simple and maybe not surprising this positive analysis is important in the literature of the field and should be published. The characteristics studied in this analysis are likely to be predictive of the value of disability benefits, hence this is consistent with the normative analysis. 

Although the strategy used to quantify the ex-ante value of USDP is not novel, the findings that these problems are valuable are incredibly relevant. Previous literature indicates that these programs increase welfare, but the public debate is still open. The quantification through the ex-ante value accounting for different states of the world is a new approach to find welfare in the context of disability programs. The policy analysis examines the cost and value of implementing alternative programs and reducing the cost of the USDP are relevant and enhance the worth of this work. 

Sufficient discussion and analysis about the measurement errors as well as robustness checks using different ways of classifying health severity make me confident that the results are solid. It also plays in its favor that the majority of biases in measurement error  I also made suggestions pertinent to the data section, the appendixes, and other minor details in the main document. More importantly, I do not find any serious concern about theory,  structure, or content that made me apprehensive about this work. Nevertheless, these small issues must be satisfactorily addressed. For all the previous reasons I recommend that this paper is accepted conditional on reviewing the changes suggested in the report. 

Respectfully,

\hspace{2cm}

Giselle Labrador Badia
(Referee)

\end{document}
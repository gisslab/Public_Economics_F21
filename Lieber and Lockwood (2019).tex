\documentclass{article}
\usepackage[utf8]{inputenc}

%\usepackage[left=3cm, right=2.5cm, top=2.5cm, bottom=2.5cm]{geometry}e}
\usepackage[spanish,english]{babel}
\usepackage{apacite}
\usepackage[round]{natbib}
\usepackage{hyperref}
\usepackage[margin = 1in, top=2cm]{geometry}% Margins
\setlength{\parindent}{2em}
\setlength{\parskip}{0.2em}
\usepackage{setspace} % Setting the spacing between lines
\usepackage{hyperref} % To create hyperlinks within the document
\spacing{1.15}

\usepackage[round]{natbib}
%\bibliographystyle{plainnat}
\bibliographystyle{apacite}

\title{Outline of Lieber and Lockwood (2019)}
\author{Giselle Labrador Badia}
\date{October 2021}

\begin{document}

\maketitle

The following is an outline of the introduction of \cite{lieber2019targeting}

\begin{enumerate}

\item Introduction: In-kind transfer make up large parts of government programs and there is debate about the desirability of flexible benefits.

\item  Motivation: Little is known about the magnitudes of the cost-benefits (trade-off) of in-kind transfers, i.e., cost of not having a cash and targeting effects. 

\item Objective: Develop an approach to quantify the trade-off of in-kind transfer and apply it to homecare.

\item Empirical Approach 1: Quantify moral hazard effect and finds that in-kind provision increases formal care consumption, suggesting that recipients value the benefit below its cost.

\item Empirical Approach 2: Estimates distribution of consumption of formal care and finds heterogeneity with suggest concentration of transfers. 

\item Empirical Approach 3: Link between consumption and the marginal utility of income, and finds that the provision concentrates transfer, and the recipients are sicker and with fewer informal caregivers. 

\item Methodology and results: combining reduced form with a structural model with a stylized expected utility framework finds substantial welfare gains over a cash-benefit contract. 

\item Contribution to the Literature 1: Reveals importance on welfare of care insurance of risk within unhealthy states of the world and moral hazard.

\item Contribution to the Literature 2: Connects theoretical and empirical literature on in-kind transfers pointing out that the key impact of recipient's cost reduction. 

\item Contribution to the Literature 3: On targeting in benefit programs results suggest that though they can cause low take-up also increase welfare through targeting. 

\end{enumerate}





\bibliography{references}

\end{document}

\documentclass{article}
\usepackage[utf8]{inputenc}

%\usepackage[left=3cm, right=2.5cm, top=2.5cm, bottom=2.5cm]{geometry}e}
\usepackage[spanish,english]{babel}
\usepackage{apacite}
\usepackage[round]{natbib}
\usepackage{hyperref}
\usepackage[margin = 1in, top=2cm]{geometry}% Margins
\setlength{\parindent}{2em}
\setlength{\parskip}{0.2em}
\usepackage{setspace} % Setting the spacing between lines
\usepackage{hyperref} % To create hyperlinks within the document
\spacing{1.2}

\usepackage[round]{natbib}
%\bibliographystyle{plainnat}
\bibliographystyle{apacite}

\title{Outline of Hendren and Sprung-Keyser (2020)}
\author{Giselle Labrador Badia}
\date{October 2021}

\begin{document}

\maketitle

The following is an outline of the introduction of \cite{hendren2020unified}

\begin{enumerate}

\item Research questions: What government expenditures are more effective at improving social welfare?

\item  Motivation: Larger empirical literature on empirical government policies uses varying measures making difficult policy comparisons.

\item Objective: Using existing literature the article conducts a comparative welfare analysis of 133 historical tax expenditures policies implemented in the US over the past half-century.

\item Empirical Approach: The marginal value of public funds (MVPF), the measure used to answer research questions, measures the welfare that can be delivered to beneficiaries per dollar of government expenditure in the policy. 

\item More on MVPF: By measuring shadow prices, MVPF measures the feasible trade-offs to the government  

\item Construction of MVPF: Uses willingness to pay (benefits) and net government costs.

\item Constraints and limitations: Scope of existing literature and omission of effects that cannot be translated to MVPF.

\item Results 1: Highest MVPFs found for direct investments in health and education of low-income children.  

\item Results 2: High MVPF in children of all ages rather than diminishing marginal return throughout childhood. 

\item Results 3: Smaller MVPF for policies targeting adults (between 0.5-2)

\item Results 4: Large variations in MVPFs across policies and within age.

\item Results 5: Among expenditures on adults, the larger MVPFs are for reductions in top marginal tax rates and policies that have spillovers on children.

\item Relation to previous literature: Estimates are related to existing theories of optimal government policies

\item Implications for future research 1: showing how the MVPF framework allows quantifying the value of research.

\item Implications for future research 2: Insights that come using MVPF as opposed to traditional cost-benefit analysis.

\item Implications for future research 3: future empirical design, in particular, the importance of determining whether willingness to pay is positive or negative. 

\item Contribution to the Literature 1: existing research making the argument for investment in children and importance of  long-run effect of safety net protections for children. 

\item Contribution to the Literature 2: tax literature and argument that government expenditures pay for themselves, finding evidence of Laffer effects for investment in children. 

\item Roadmap

\end{enumerate}





\bibliography{references}

\end{document}

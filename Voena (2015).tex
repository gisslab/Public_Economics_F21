\documentclass{article}
\usepackage[utf8]{inputenc}

%\usepackage[left=3cm, right=2.5cm, top=2.5cm, bottom=2.5cm]{geometry}e}
\usepackage[spanish,english]{babel}
\usepackage{apacite}
\usepackage[round]{natbib}
\usepackage{hyperref}
\usepackage[margin = 1in, top=2cm]{geometry}% Margins
\setlength{\parindent}{2em}
\setlength{\parskip}{0.2em}
\usepackage{setspace} % Setting the spacing between lines
\usepackage{hyperref} % To create hyperlinks within the document
\spacing{1.2}

\usepackage[round]{natbib}
%\bibliographystyle{plainnat}
\bibliographystyle{apacite}

\title{Outline of Voena (2015)}
\author{Giselle Labrador Badia}
\date{October 2021}

\begin{document}

\maketitle

The following is an outline of the introduction of \cite{voena2015yours}

\begin{enumerate}

\item Objective 1:  Examines how divorce laws influence intertemporal behavior and the well-being of couples.

\item Objective 2: Build a dynamic model of household decisions to understand welfare implications.

\item Data and Results: The introduction of unilateral divorce leads to higher accumulation of assets and married women are less likely to work. 

\item Mechanism: property division laws affect spouse's divorce allocation which affects the intra-household allocation during marriage. 

\item Empirical Approach 1: Identity parameters of intra-household allocation, compute welfare effects of reforms, and perform counterfactual experiments. 

\item Empirical approach 2: Estimating by indirect inference, the model replicates the responses of assets accumulation and female employment if wives Pareto weight is lower. 

\item Results 2: Women have a lower share of the couples' assets and benefit from the laws obtaining on average more consumption during marriage and after divorce.

\item Discussion of results: Asset accumulation during marriage increases because spouses' individual incentives to save are distorted by reforms (equivalent to a tax on savings or subsidy).

\item Contribution 1: Develop and estimate a dynamic model that incorporates mutual consent versus unilateral divorce regimes and property division low. 

\item Contribution 2: Documents and explains the empirical relationship between changes in divorce laws and the saving behavior of married couples.

\item Contribution 3: Implications of the current US property division laws on couples' welfare and policy implications. 

\end{enumerate}



\bibliography{references}

\end{document}

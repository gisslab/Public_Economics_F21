\documentclass{article}
\usepackage[utf8]{inputenc}

%\usepackage[left=3cm, right=2.5cm, top=2.5cm, bottom=2.5cm]{geometry}e}
\usepackage[spanish,english]{babel}
\usepackage{apacite}
\usepackage[round]{natbib}
\usepackage{hyperref}
\usepackage[margin = 1in, top=2cm]{geometry}% Margins
\setlength{\parindent}{2em}
\setlength{\parskip}{0.2em}
\usepackage{setspace} % Setting the spacing between lines
\usepackage{hyperref} % To create hyperlinks within the document
\spacing{1.05}

\usepackage[round]{natbib}
%\bibliographystyle{plainnat}
\bibliographystyle{apacite}

\title{Outline of Maestas et al. (2013)}
\author{Giselle Labrador Badia}
\date{September 2021}

\begin{document}

\maketitle

The following is an outline of the introduction of \cite{maestas2013does}

\begin{enumerate}

\item Introduction: US disability program expenditures have increased dramatically over the last decades, and it was at the moment (2013) an immediate problem.
\item Motivation: Although the Social Security Disability Insurance (SSDI) caseload has grown, the employment of disabled workers has steadily declined.
\item Research literature consensus: The structure of the SSDI program is a major determinant in the decline in employment and associated program growth.
\item Contribution 1: First casual estimate of the effect of SSDI receipt on labor supply using the population of SSDI applicants, an exogenous variation.
\item Literature: Early analysis (\cite{parsons1980decline}) attributed the entire fall in employment to the SSDI program and  \cite{bound1989health} first proposed using denied applicants as control groups and an upper bound on the total program effect.
\item Literature: Parson replies to Bound arguing that he understated the employment potential of SSDI because the application process also reduces the labor supply of rejected applicants.
\item Literature: \cite{von2011trends} find using appeal outcomes that the SSDI recipients employment potential had risen mainly because of nonterminal impairments (e.g. mental health), though they lack an exogenous variation.

\item Literature: Other studies improved on the previous approach by using policy variations in initial allowance rates to alleviate the confounding with lifetime earnings and impairment severity (see \cite{gruber1997disability})

\item Contribution 1: The development of a research design to estimate the causal effect of SSDI receipt on labor supply along the margins using examiner's allowance propensity as an instrumental variable.

\item Finding: Employment rate of beneficiaries on the margin of entry in 2005 and 2006 would have been on average 28 percent higher two years later if they have not received benefits.

\item Comparing findings with literature: \cite{french2014effect} finds using variations in the propensity of law judges that the employment rate of applicants denied at the hearing level would have been 26 percent higher.

\item Contribution 2: Estimation and characterization of SSDI applicants in the margin of program entry concluded that these make up 23 percent of applicants that tend to be younger, poorer and have mental impairments.


\item Contribution 3: The continuous instrumental variable allows to test for heterogeneous treatments effects and estimate this distribution.



\end{enumerate}

\bibliography{references}


\end{document}
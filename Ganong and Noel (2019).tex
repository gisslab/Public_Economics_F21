\documentclass{article}
\usepackage[utf8]{inputenc}

%\usepackage[left=3cm, right=2.5cm, top=2.5cm, bottom=2.5cm]{geometry}e}
\usepackage[spanish,english]{babel}
\usepackage{apacite}
\usepackage[round]{natbib}
\usepackage{hyperref}
\usepackage[margin = 1in, top=2cm]{geometry}% Margins
\setlength{\parindent}{2em}
\setlength{\parskip}{0.2em}
\usepackage{setspace} % Setting the spacing between lines
\usepackage{hyperref} % To create hyperlinks within the document
\spacing{1.05}

\usepackage[round]{natbib}
%\bibliographystyle{plainnat}
\bibliographystyle{apacite}

\title{Outline of Ganong and Noel (2019) }
\author{Giselle Labrador Badia}
\date{September 2021}

\begin{document}

\maketitle

This is an outline of the introduction of \cite{ganong2019consumer}.

\begin{enumerate}
    \item Establish that the goal of the paper is to document the monthly path of spending during employment and assess implications.
    \item State main findings of the paper: spending is highly sensitive to income, test to distinguish theories to explain this sensitivity and UI policy implications. 
    \item Describe the de-identify monthly panel data.
    \item Discuss empirical finding number 1: spending drops at UI benefits exhaustion.
    \item Discuss empirical finding number 2: Spending drops sharply on necessities such as groceries.
    \item Discuss empirical finding number 3:  variations in income cause the changes in spending and Florida as an out-of-sample test.
    \item  Continues with New Jersey as an out-of-sample test.
    \item Discuss empirical finding number 4: job-finding spikes modestly at UI benefit
    \item Outline the novel test to distinguish theories (rational and behavioral) about income sensitivity. 
    \item Emphasize the importance of the test given that theories have different predictions and give away that rational models cannot explain well the path of spending.
    \item Propose the path of spending around a predictable income decline as identified moment. 
    \item Give reasons why spending of unemployed is useful to study consumption models: ubiquitous and no time available for home production.
    \item Formally describe the test with a structural
model of consumption and job search and various parametrizations and contribution to the literature by studying these effects in a quantitative structural model.
\item Test result: Behavioral models with present biased or myopic agents match the data much better.
\item Normative implications and contribution: fist estimate of drop and measure of the average of UI benefit
\item Contribution to the literature: estimate the consumption smoothing gains from extending the duration
of UI benefits.
\item Normative finding: the welfare gains from improved consumption-smoothing due to
extending the duration of UI benefits are four times as large as from raising the level
of UI benefits 
\item Roadmap
\end{enumerate}

\bibliography{references}

\end{document}

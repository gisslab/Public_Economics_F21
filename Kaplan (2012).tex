\documentclass{article}
\usepackage[utf8]{inputenc}

%\usepackage[left=3cm, right=2.5cm, top=2.5cm, bottom=2.5cm]{geometry}e}
\usepackage[spanish,english]{babel}
\usepackage{apacite}
\usepackage[round]{natbib}
\usepackage{hyperref}
\usepackage[margin = 1in, top=2cm]{geometry}% Margins
\setlength{\parindent}{2em}
\setlength{\parskip}{0.2em}
\usepackage{setspace} % Setting the spacing between lines
\usepackage{hyperref} % To create hyperlinks within the document
\spacing{1.05}

\usepackage[round]{natbib}
%\bibliographystyle{plainnat}
\bibliographystyle{apacite}

\title{Outline of Kaplan (2012) }
\author{Giselle Labrador Badia}
\date{September 2021}

\begin{document}

\maketitle

The following is an outline of the introduction of \cite{kaplan2012moving}.

\begin{enumerate}
    \item Motivated by the absent possibility of parental coresidence in the literature of labor market insurance, the authors use an estimated structural model to show that this option is a channel of insurance against labor market risks. 
    \item Explains main features of the model, such as the dynamic game that determines parental coresidence.
    \item Explains two types of idiosyncratic uncertainty, shocks to the labor market and to youth's desires to live with parents (preference shock).
    \item Discuss finding: market labor shocks affect where and when students move with their parents and preference shocks affect the cross-sectional difference in living arrangements which implicates the importance of the dynamic structural model. 
    \item Empirical finding: estimated model is used to measure the value of the insurance, and finds that parental coresidence is valuable for all youths, especially the poor.  
    \item Finding through counterfactual exercises 1: the option to move in and out decreases consumption drop in response to a job loss.
    \item Finding through counterfactual exercises 2:  the possibility of parental coresidence lowers the job acceptance probability, which is consistent with a high aggregate labor elasticity in young workers. 
    \item Finding through counterfactual exercises 3:  The option of moving back home decreases asset accumulation. 
    \item Finding through counterfactual exercises 4:  The option to live at home has a positive and substantial impact on youth's future earnings because it facilitates the pursuit of jobs.
    \item Empirical evidence of the prediction about impact in future earnings.
    \item Empirical finding: This shows that the relationship between parent-child living arrangements and the labor market extends to older individuals.
 \item Contribution of empirical approach: analysis with high-frequency data allows for the estimation of duration models  and permits the construction of counterfactuals to study the impact of coresidence 

\end{enumerate}

\bibliography{references}

\end{document}

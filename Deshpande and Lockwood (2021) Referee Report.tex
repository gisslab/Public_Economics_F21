\documentclass{article}
\usepackage[utf8]{inputenc}

%\usepackage[left=3cm, right=2.5cm, top=2.5cm, bottom=2.5cm]{geometry}e}
\usepackage[spanish,english]{babel}
\usepackage{apacite}
\usepackage[round]{natbib}
\usepackage{hyperref}
\usepackage[margin = 1.35in, top=2cm]{geometry}% Margins
\setlength{\parindent}{2.0em}
\setlength{\parskip}{0.3em}
\usepackage{setspace} % Setting the spacing between lines
\usepackage{hyperref} % To create hyperlinks within the document
\spacing{1.25}

\usepackage[round]{natbib}
%\bibliographystyle{plainnat}
\bibliographystyle{apacite}

\title{Report on 'Beyond health: Non-health risk and the value of disability insurance'}
\author{Anonymous referee}
\date{December 2022}

\begin{document}

\maketitle

This paper speaks to the public debate about the value of disability programs and whether they should be focused only on individuals with severe health conditions. To answer this question, the authors quantify the overall insurance value of the U.S. disability program, including value from insuring non-health risks (like productivity shocks, job loss, eviction, etc). More precisely, the authors use positive and normative analysis to respond to which extent mismatches concerning health insure or aggravate different risks. As part of the positive analysis, they find that disability recipients, in particular, those without severe health risks are worse off than non-recipients.  The normative analysis establishes how the disability programs including the benefits to less-severe recipients turn into ex-ante value, which is determined by the marginal utility of income and "counterfactual earnings". They find that benefits to recipients without severe health issues have an annual surplus. This result suggests that insuring less-severe health conditions is not diluting the value of U.S disability programs but boosting it. The explanation given by the authors is that favorable selection into the programs is so strong, that less-severe recipients are worse off than more-severe cases (including more-severe recipients) so the value of USDP is enhanced.

The following are comments both about the strength and weaknesses of the paper, as well as suggestions for improvement. 
\begin{enumerate}

    \item  The papers motivate adequately the need to look beyond health given the cost and enormous impact that the disability programs have.  The argument that risks should be insured by their own programs is naive given the complex nature of non-health risks. Thus, the authors do a good job by letting the welfare analysis be weighted against non-welfarist analysis that calls out for focus on targeting severity of health issues but for which quantitative evidence is limited. 
    
    \item Although the references are extensive, the literature review feels incomplete in the sense that is mostly in line with the author's findings. For example, there is no mention of any previous work that indicates that the value of disability programs could be improved by reducing in some way the size or by changing the aim of the programs. Only the public debate attention to these issues is mentioned. A reference of this kind will make the paper more relevant and will explain why the debate is not settled in the academic arena. Also, the authors point out that the evidence about the non-welfarist planner is poor but does not cite any work within this literature. 


    \item In section 2 (Theory), is not clear what is Figure 1 showing: are these points inspired by the data or just random points to illustrate the different perspectives? The authors wrote that the "blue points are a hypothetical joint". It will be nice if the distributions of points had some relation with what is observed in the data.  Many more points are located in the Southwest quadrant (No health shocks, negative surplus); why did they choose this distribution?. The authors could clarify and make explicit exactly what does this means in terms of health as a signal of surplus. %One explanation is that this graph includes non-recipients which are the largest group and mainly composed of less-severe cases in which case makes sense that the surplus is negative. In any case, the authors could clarify and make explicit where do the plotted points come from.
    
    \item The data description and discussion are too short in the main document. I think that this section should at least mention the size of each data used and discuss in more detail the different data sources used. These elements are clear and extensive in the Appendix, but I believe that some important features of the data should be included in Section 3. For example, the fact the PSID is small and an explanation for why other data sets were also need it. The current version of the paper only mentions the advantages of this data source and not the shortcomings. 
    
    
    
    \item The facts discovered in the positive analysis do not seem surprising. Especially fact 2, since the disability programs have earning limits and it is logical that individuals seeking the receipt will be worse off than individuals that are not. Nonetheless, the authors touch on these subjects and talk about the selection process into the programs.  The positive analysis is also simple and there are no associations between variables, it is mostly descriptive statistics of the groups.  However, if these facts have not been noticed before in the disability insurance literature, it is important that these are highlighted. The characterization of more-severe and less-severe disability recipients and non-recipients is an important contribution to the literature.  Moreover, these descriptive comparisons serve as a motivation and are coherent with the subsequent normative analysis. 
    
    
    \item The normative analysis satisfactory estimates disability benefits as being more valuable than a tax cut with the same cost. Aside from some small variations and robustness calculations of counterfactual earnings, marginal utility focus on the tax cut approach to cost. The analysis evaluates the ex-ante value and different states of the natures which I think is a good feature of the quantitative part. The theory and model help to shed light on the mechanism underlying the welfare impact of disability programs. % It uses a baseline fix benefit of the annual cash benefit of disability recipients.  
    
    \item Although selective application into the programs is presented as one of the main drivers of the overall value of the USDP, the features of this selection that lead to the increase in value could be studied in depth. The investigation reduces to two hypothetical programs, the first one is a random acceptance program using base distributions of success while the second one is based on an earning limit. It is not surprising that the program that focuses on earning has far better results in terms of surplus. Analogous hypothetical programs for different risks and sets of non-health risks as well as a marginal utility could be implemented. 
    
    \item Robustness checks on the measures of severity although it cannot be quantified are discussed thoroughly for Facts 2, 3, and the normative analysis.  It might be worth considering the inclusion of comment or discussion about the possible measurement error of Fact 1, or why there is no need for it. 
    
  %  \item   iT MIGHT ALSO BE WORTH CONSIDERING
  % IT HAS TO BE ACKNOWLEDGE
\end{enumerate}






%\bibliography{references}

\end{document}
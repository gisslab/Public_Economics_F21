\documentclass{article}
\usepackage[utf8]{inputenc}

%\usepackage[left=3cm, right=2.5cm, top=2.5cm, bottom=2.5cm]{geometry}e}
\usepackage[spanish,english]{babel}
\usepackage{apacite}
\usepackage[round]{natbib}
\usepackage{hyperref}
\usepackage[margin = 1in, top=2cm]{geometry}% Margins
\setlength{\parindent}{2em}
\setlength{\parskip}{0.2em}
\usepackage{setspace} % Setting the spacing between lines
\usepackage{hyperref} % To create hyperlinks within the document
\spacing{1.15}

\usepackage[round]{natbib}
%\bibliographystyle{plainnat}
\bibliographystyle{apacite}

\title{Outline of Low and Pistaferri (2015)}
\author{Giselle Labrador Badia}
\date{September 2021}

\begin{document}

\maketitle

The following is an outline of the introduction of \cite{low2015disability}

\begin{enumerate}
\item Motivation and key research questions: are those unable to work and in need receiving benefits?, and how valuable is the growing ID program?

\item Objective: To provide a  framework to evaluate welfare consequences of reforms and compare insurance and disincentive effect of disability 
\item To address the aims, the authors propose a life-cycle framework to estimate, and study welfare effects on the behavior of policies parameters like generosity, screening process, alternative programs and reassessment rate. 
\item Findings: DI insurance programs have substantial false rejection rates, but fewer acceptance rates; and welfare increases if the threshold for acceptance is lower, disability payments are higher, reassessment less frequently and food stamp payments more generous. 
\item Advantage of structural model and key characteristics of the model: the possibility of evaluating novel policy reforms and counterfactual cases, and robustness of the model proved by various validity tests. 
\item Roadmap


\end{enumerate}

\bibliography{references}


\end{document}